\usepackage{amsmath, amsfonts, amssymb}
\usetheme{metropolis}
\usepackage{FiraSans}

% Japanese support
\usepackage[utf8]{inputenc}
\usepackage{xeCJK}
\setCJKmainfont{IPAexMincho}
\setCJKsansfont{IPAexGothic}
\setCJKmonofont{IPAexGothic}
% \setmainfont{ipamp.ttf}
% \setsansfont{ipagp.ttf}
% \setmonofont{ipag.ttf}

% Add TOC at the beginning of each section
\AtBeginSection[]{\begin{frame}
        \frametitle{Outline}
        \tableofcontents[currentsection, currentsubsection, subsectionstyle=show/shaded/hide]
    \end{frame}
}

% Toggle text font between Sans Serif & Serif (math mode font unchanged)
%   \renewcommand{\familydefault}{}           # sans serif
%   \renewcommand{\familydefault}{\sfdefault} # serif
\renewcommand{\familydefault}{\sfdefault}

% Dummy command to enable markdown inside raw latex
% example:
%   \tex{
%     \begin{center}
%   }
%   Markdown *here*.
%   \tex{
%     \end{center}
%   }
\newcommand\tex[1]{#1}

% useful symbols
\newcommand{\heart}{\ensuremath\heartsuit}
\newcommand{\coheart}{\rotatebox[origin=c]{180}{\heart}}

\newcommand{\dialabel[1]}{\ensuremath\langle#1\rangle}
\newcommand{\boxlabel[1]}{\ensuremath[#1]}

\definecolor{DeepBlue}{RGB}{0, 0, 78}
\setbeamercolor{title}{fg=DeepBlue}
\setbeamercolor{frametitle}{bg=DeepBlue}
\setbeamercolor{structure}{bg=DeepBlue}
